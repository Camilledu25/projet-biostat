% Options for packages loaded elsewhere
\PassOptionsToPackage{unicode}{hyperref}
\PassOptionsToPackage{hyphens}{url}
%
\documentclass[
]{article}
\usepackage{lmodern}
\usepackage{amssymb,amsmath}
\usepackage{ifxetex,ifluatex}
\ifnum 0\ifxetex 1\fi\ifluatex 1\fi=0 % if pdftex
  \usepackage[T1]{fontenc}
  \usepackage[utf8]{inputenc}
  \usepackage{textcomp} % provide euro and other symbols
\else % if luatex or xetex
  \usepackage{unicode-math}
  \defaultfontfeatures{Scale=MatchLowercase}
  \defaultfontfeatures[\rmfamily]{Ligatures=TeX,Scale=1}
\fi
% Use upquote if available, for straight quotes in verbatim environments
\IfFileExists{upquote.sty}{\usepackage{upquote}}{}
\IfFileExists{microtype.sty}{% use microtype if available
  \usepackage[]{microtype}
  \UseMicrotypeSet[protrusion]{basicmath} % disable protrusion for tt fonts
}{}
\makeatletter
\@ifundefined{KOMAClassName}{% if non-KOMA class
  \IfFileExists{parskip.sty}{%
    \usepackage{parskip}
  }{% else
    \setlength{\parindent}{0pt}
    \setlength{\parskip}{6pt plus 2pt minus 1pt}}
}{% if KOMA class
  \KOMAoptions{parskip=half}}
\makeatother
\usepackage{xcolor}
\IfFileExists{xurl.sty}{\usepackage{xurl}}{} % add URL line breaks if available
\IfFileExists{bookmark.sty}{\usepackage{bookmark}}{\usepackage{hyperref}}
\hypersetup{
  pdftitle={biostat},
  pdfauthor={camille mathilde},
  hidelinks,
  pdfcreator={LaTeX via pandoc}}
\urlstyle{same} % disable monospaced font for URLs
\usepackage[margin=1in]{geometry}
\usepackage{color}
\usepackage{fancyvrb}
\newcommand{\VerbBar}{|}
\newcommand{\VERB}{\Verb[commandchars=\\\{\}]}
\DefineVerbatimEnvironment{Highlighting}{Verbatim}{commandchars=\\\{\}}
% Add ',fontsize=\small' for more characters per line
\usepackage{framed}
\definecolor{shadecolor}{RGB}{248,248,248}
\newenvironment{Shaded}{\begin{snugshade}}{\end{snugshade}}
\newcommand{\AlertTok}[1]{\textcolor[rgb]{0.94,0.16,0.16}{#1}}
\newcommand{\AnnotationTok}[1]{\textcolor[rgb]{0.56,0.35,0.01}{\textbf{\textit{#1}}}}
\newcommand{\AttributeTok}[1]{\textcolor[rgb]{0.77,0.63,0.00}{#1}}
\newcommand{\BaseNTok}[1]{\textcolor[rgb]{0.00,0.00,0.81}{#1}}
\newcommand{\BuiltInTok}[1]{#1}
\newcommand{\CharTok}[1]{\textcolor[rgb]{0.31,0.60,0.02}{#1}}
\newcommand{\CommentTok}[1]{\textcolor[rgb]{0.56,0.35,0.01}{\textit{#1}}}
\newcommand{\CommentVarTok}[1]{\textcolor[rgb]{0.56,0.35,0.01}{\textbf{\textit{#1}}}}
\newcommand{\ConstantTok}[1]{\textcolor[rgb]{0.00,0.00,0.00}{#1}}
\newcommand{\ControlFlowTok}[1]{\textcolor[rgb]{0.13,0.29,0.53}{\textbf{#1}}}
\newcommand{\DataTypeTok}[1]{\textcolor[rgb]{0.13,0.29,0.53}{#1}}
\newcommand{\DecValTok}[1]{\textcolor[rgb]{0.00,0.00,0.81}{#1}}
\newcommand{\DocumentationTok}[1]{\textcolor[rgb]{0.56,0.35,0.01}{\textbf{\textit{#1}}}}
\newcommand{\ErrorTok}[1]{\textcolor[rgb]{0.64,0.00,0.00}{\textbf{#1}}}
\newcommand{\ExtensionTok}[1]{#1}
\newcommand{\FloatTok}[1]{\textcolor[rgb]{0.00,0.00,0.81}{#1}}
\newcommand{\FunctionTok}[1]{\textcolor[rgb]{0.00,0.00,0.00}{#1}}
\newcommand{\ImportTok}[1]{#1}
\newcommand{\InformationTok}[1]{\textcolor[rgb]{0.56,0.35,0.01}{\textbf{\textit{#1}}}}
\newcommand{\KeywordTok}[1]{\textcolor[rgb]{0.13,0.29,0.53}{\textbf{#1}}}
\newcommand{\NormalTok}[1]{#1}
\newcommand{\OperatorTok}[1]{\textcolor[rgb]{0.81,0.36,0.00}{\textbf{#1}}}
\newcommand{\OtherTok}[1]{\textcolor[rgb]{0.56,0.35,0.01}{#1}}
\newcommand{\PreprocessorTok}[1]{\textcolor[rgb]{0.56,0.35,0.01}{\textit{#1}}}
\newcommand{\RegionMarkerTok}[1]{#1}
\newcommand{\SpecialCharTok}[1]{\textcolor[rgb]{0.00,0.00,0.00}{#1}}
\newcommand{\SpecialStringTok}[1]{\textcolor[rgb]{0.31,0.60,0.02}{#1}}
\newcommand{\StringTok}[1]{\textcolor[rgb]{0.31,0.60,0.02}{#1}}
\newcommand{\VariableTok}[1]{\textcolor[rgb]{0.00,0.00,0.00}{#1}}
\newcommand{\VerbatimStringTok}[1]{\textcolor[rgb]{0.31,0.60,0.02}{#1}}
\newcommand{\WarningTok}[1]{\textcolor[rgb]{0.56,0.35,0.01}{\textbf{\textit{#1}}}}
\usepackage{longtable,booktabs}
% Correct order of tables after \paragraph or \subparagraph
\usepackage{etoolbox}
\makeatletter
\patchcmd\longtable{\par}{\if@noskipsec\mbox{}\fi\par}{}{}
\makeatother
% Allow footnotes in longtable head/foot
\IfFileExists{footnotehyper.sty}{\usepackage{footnotehyper}}{\usepackage{footnote}}
\makesavenoteenv{longtable}
\usepackage{graphicx,grffile}
\makeatletter
\def\maxwidth{\ifdim\Gin@nat@width>\linewidth\linewidth\else\Gin@nat@width\fi}
\def\maxheight{\ifdim\Gin@nat@height>\textheight\textheight\else\Gin@nat@height\fi}
\makeatother
% Scale images if necessary, so that they will not overflow the page
% margins by default, and it is still possible to overwrite the defaults
% using explicit options in \includegraphics[width, height, ...]{}
\setkeys{Gin}{width=\maxwidth,height=\maxheight,keepaspectratio}
% Set default figure placement to htbp
\makeatletter
\def\fps@figure{htbp}
\makeatother
\setlength{\emergencystretch}{3em} % prevent overfull lines
\providecommand{\tightlist}{%
  \setlength{\itemsep}{0pt}\setlength{\parskip}{0pt}}
\setcounter{secnumdepth}{-\maxdimen} % remove section numbering
\usepackage{booktabs}
\usepackage{longtable}
\usepackage{array}
\usepackage{multirow}
\usepackage{wrapfig}
\usepackage{float}
\usepackage{colortbl}
\usepackage{pdflscape}
\usepackage{tabu}
\usepackage{threeparttable}
\usepackage{threeparttablex}
\usepackage[normalem]{ulem}
\usepackage{makecell}
\usepackage{xcolor}

\title{biostat}
\author{camille mathilde}
\date{22/12/2020}

\begin{document}
\maketitle

\newpage

\hypertarget{introduction}{%
\section{Introduction}\label{introduction}}

De nos jours, les maladies sont de plus en plus étudiées et de mieux en
mieux comprises. De nombreux organismes et instituts recherchent des
solutions pour vaincre ces maladies en trouvant des traitements. De
nombreuses diciplines sont impliquées, notamment la biostatistique, qui
permet d'étudier différentes situations avec des données. Dans notre
cas, nous allons étudier l'occurence des maladies coronarienne du coeur.
Une étude préalable à déja été faite sur une cohorte d'individus. Ces
individus ont répondu à une date donnée à une enquête sur les habitudes
alimentaires. Les réponses ont été recueilli dans une base de donnée que
l'on nomme \textbf{Coeur}.\\
L'objectif de cette étude sera de connaître les bonnes habitudes
alimentaires à adopter pour se prevenir d'une maladie coronarienne.
\newpage \tableofcontents \newpage \#\# Présentation de la base de
donnée

Notre table de données contient 337 individus et 15 variables qui sont

\begin{itemize}
\item \textbf{Id} : l'identifiant du sujet. \
\item   \textbf{DateEntrée} et \textbf{Date de sortie} : les dates d’entrée et de sortie de l’étude. \
\item   \textbf{Date Naissance} : la date de naissance. \
\item   \textbf{Statut} : si la sortie de l’enquête est due à une maladie coronarienne de cœur, alors le type de maladie est indiqué (la signification du code n’est pas précisée ici). Si l’individu est sain à la sortie  de l’enquête, alors le code vaut 0.\
\item   \textbf{Emploi} : le type d’emploi.\
\item   \textbf{MoisEnquête} : le mois (1= Janvier, 12= Décembre) où l’individu a répondu à l’enquête sur ses pratiques alimentaires.\
\item   \textbf{Taille/Poids} : la taille et le poids de l'individu (en cm et en kg).\
\item   \textbf{Graisse} : la quantité moyenne de graisse ingérée par jour (g/jour).\
\item   \textbf{Fibres} : la quantité moyenne de fibres ingérée par jour (g/jour).\
\item   \textbf{Consommation} : la quantité de calories(/100) ingérée par jour.\
\item   \textbf{hauteConsomation} : une variable binaire, recodage de la variable consommation.\
\item   \textbf{MCC} : une variable binaire, recodage de la variable statut (1=MCC, 0=pas de MCC). \
\end{itemize}

Nous avons également rajouté la variable \(\textbf{IMC}\) en divisant le
poids par la taille au carrée pour faire un lien entre la condition
physique et la maladie du coeur car le poids ou la taille tout seul ne
suffisent pas pour savoir si une personne est en bonne santé ou en
surpoids.\\

Nous nous sommes également rendu compte qu'il y avait des erreurs dans
le recodage de la variables \textbf{statut}. En effet, certains
individus avaient contracté une maladie du coeur, mais la variable
recodage \textbf{MCC} ne l'avait pas pris en compte nous avons donc
rectifié ça. Nous avons aussi remarqué que certaines variables
qualitatives étaient en \textbf{numeric}, ce qui posera problème pour
notre étude. Nous les recodons donc en \textbf{factor}.\\
Nous observons enfin que la table de données contient des valeurs
manquantes. Nous enlevons donc chaque ligne qui contient au moins une
valeur manquante. Nous passons donc de 337 à 328 individus.\\

Regardons les 5 premières lignes de notre table de données :\\

\begin{Shaded}
\begin{Highlighting}[]
\NormalTok{coeur <-}\StringTok{ }\KeywordTok{readRDS}\NormalTok{(}\StringTok{'data/my_data_frame.rds'}\NormalTok{)}
\NormalTok{coeur <-coeur}\OperatorTok\KeywordTok{drop_na}\NormalTok{()}
\NormalTok{coeur <-coeur}\OperatorTok\NormalTok{dplyr}\OperatorTok{::}\KeywordTok{select}\NormalTok{(}\OperatorTok{-}\NormalTok{X1)}
\NormalTok{coeur}\OperatorTok{$}\NormalTok{statut<-}\KeywordTok{as.factor}\NormalTok{(coeur}\OperatorTok{$}\NormalTok{statut)}
\NormalTok{coeur}\OperatorTok{$}\NormalTok{emploi<-}\KeywordTok{as.factor}\NormalTok{(coeur}\OperatorTok{$}\NormalTok{emploi)}
\NormalTok{coeur}\OperatorTok{$}\NormalTok{moisEnqu_e<-}\KeywordTok{as.factor}\NormalTok{(coeur}\OperatorTok{$}\NormalTok{moisEnqu_e)}
\NormalTok{coeur}\OperatorTok{$}\NormalTok{hauteConsomation<-}\KeywordTok{as.factor}\NormalTok{(coeur}\OperatorTok{$}\NormalTok{hauteConsomation)}

\NormalTok{coeur<-}\KeywordTok{mutate}\NormalTok{(coeur,}\DataTypeTok{imc =}\NormalTok{poids }\OperatorTok{/}\NormalTok{(taille}\OperatorTok{/}\DecValTok{100}\NormalTok{)}\OperatorTok{^}\DecValTok{2}\NormalTok{)}
\NormalTok{coeur<-coeur }\OperatorTok\StringTok{ }\NormalTok{dplyr}\OperatorTok{::}\KeywordTok{select}\NormalTok{(}\OperatorTok{-}\NormalTok{MCC)}
\NormalTok{coeur <-}\KeywordTok{mutate}\NormalTok{(coeur,}\DataTypeTok{MCC =}\KeywordTok{case_when}\NormalTok{(}
\NormalTok{  statut}\OperatorTok{!=}\DecValTok{0}\OperatorTok{~}\DecValTok{1}\NormalTok{,}
  \OtherTok{TRUE}\OperatorTok{~}\DecValTok{0}\NormalTok{))}

\KeywordTok{pander}\NormalTok{(}\KeywordTok{head}\NormalTok{(coeur))}
\end{Highlighting}
\end{Shaded}

\begin{longtable}[]{@{}cccccc@{}}
\caption{Table continues below}\tabularnewline
\toprule
\begin{minipage}[b]{0.07\columnwidth}\centering
id\strut
\end{minipage} & \begin{minipage}[b]{0.15\columnwidth}\centering
dateEntree\strut
\end{minipage} & \begin{minipage}[b]{0.15\columnwidth}\centering
dateSortie\strut
\end{minipage} & \begin{minipage}[b]{0.19\columnwidth}\centering
dateNaissance\strut
\end{minipage} & \begin{minipage}[b]{0.10\columnwidth}\centering
statut\strut
\end{minipage} & \begin{minipage}[b]{0.16\columnwidth}\centering
emploi\strut
\end{minipage}\tabularnewline
\midrule
\endfirsthead
\toprule
\begin{minipage}[b]{0.07\columnwidth}\centering
id\strut
\end{minipage} & \begin{minipage}[b]{0.15\columnwidth}\centering
dateEntree\strut
\end{minipage} & \begin{minipage}[b]{0.15\columnwidth}\centering
dateSortie\strut
\end{minipage} & \begin{minipage}[b]{0.19\columnwidth}\centering
dateNaissance\strut
\end{minipage} & \begin{minipage}[b]{0.10\columnwidth}\centering
statut\strut
\end{minipage} & \begin{minipage}[b]{0.16\columnwidth}\centering
emploi\strut
\end{minipage}\tabularnewline
\midrule
\endhead
\begin{minipage}[t]{0.07\columnwidth}\centering
102\strut
\end{minipage} & \begin{minipage}[t]{0.15\columnwidth}\centering
17/01/76\strut
\end{minipage} & \begin{minipage}[t]{0.15\columnwidth}\centering
02/12/86\strut
\end{minipage} & \begin{minipage}[t]{0.19\columnwidth}\centering
02/03/39\strut
\end{minipage} & \begin{minipage}[t]{0.10\columnwidth}\centering
0\strut
\end{minipage} & \begin{minipage}[t]{0.16\columnwidth}\centering
Driver\strut
\end{minipage}\tabularnewline
\begin{minipage}[t]{0.07\columnwidth}\centering
59\strut
\end{minipage} & \begin{minipage}[t]{0.15\columnwidth}\centering
16/07/73\strut
\end{minipage} & \begin{minipage}[t]{0.15\columnwidth}\centering
05/07/82\strut
\end{minipage} & \begin{minipage}[t]{0.19\columnwidth}\centering
05/07/12\strut
\end{minipage} & \begin{minipage}[t]{0.10\columnwidth}\centering
0\strut
\end{minipage} & \begin{minipage}[t]{0.16\columnwidth}\centering
Driver\strut
\end{minipage}\tabularnewline
\begin{minipage}[t]{0.07\columnwidth}\centering
126\strut
\end{minipage} & \begin{minipage}[t]{0.15\columnwidth}\centering
17/03/70\strut
\end{minipage} & \begin{minipage}[t]{0.15\columnwidth}\centering
20/03/84\strut
\end{minipage} & \begin{minipage}[t]{0.19\columnwidth}\centering
24/12/19\strut
\end{minipage} & \begin{minipage}[t]{0.10\columnwidth}\centering
13\strut
\end{minipage} & \begin{minipage}[t]{0.16\columnwidth}\centering
Conductor\strut
\end{minipage}\tabularnewline
\begin{minipage}[t]{0.07\columnwidth}\centering
16\strut
\end{minipage} & \begin{minipage}[t]{0.15\columnwidth}\centering
16/05/69\strut
\end{minipage} & \begin{minipage}[t]{0.15\columnwidth}\centering
31/12/69\strut
\end{minipage} & \begin{minipage}[t]{0.19\columnwidth}\centering
17/09/06\strut
\end{minipage} & \begin{minipage}[t]{0.10\columnwidth}\centering
3\strut
\end{minipage} & \begin{minipage}[t]{0.16\columnwidth}\centering
Driver\strut
\end{minipage}\tabularnewline
\begin{minipage}[t]{0.07\columnwidth}\centering
247\strut
\end{minipage} & \begin{minipage}[t]{0.15\columnwidth}\centering
16/03/68\strut
\end{minipage} & \begin{minipage}[t]{0.15\columnwidth}\centering
25/06/79\strut
\end{minipage} & \begin{minipage}[t]{0.19\columnwidth}\centering
10/07/18\strut
\end{minipage} & \begin{minipage}[t]{0.10\columnwidth}\centering
13\strut
\end{minipage} & \begin{minipage}[t]{0.16\columnwidth}\centering
Bank worker\strut
\end{minipage}\tabularnewline
\begin{minipage}[t]{0.07\columnwidth}\centering
272\strut
\end{minipage} & \begin{minipage}[t]{0.15\columnwidth}\centering
16/03/69\strut
\end{minipage} & \begin{minipage}[t]{0.15\columnwidth}\centering
13/12/73\strut
\end{minipage} & \begin{minipage}[t]{0.19\columnwidth}\centering
06/03/20\strut
\end{minipage} & \begin{minipage}[t]{0.10\columnwidth}\centering
3\strut
\end{minipage} & \begin{minipage}[t]{0.16\columnwidth}\centering
Bank worker\strut
\end{minipage}\tabularnewline
\bottomrule
\end{longtable}

\begin{longtable}[]{@{}cccccc@{}}
\caption{Table continues below}\tabularnewline
\toprule
\begin{minipage}[b]{0.15\columnwidth}\centering
moisEnqu\_e\strut
\end{minipage} & \begin{minipage}[b]{0.17\columnwidth}\centering
consommation\strut
\end{minipage} & \begin{minipage}[b]{0.10\columnwidth}\centering
taille\strut
\end{minipage} & \begin{minipage}[b]{0.09\columnwidth}\centering
poids\strut
\end{minipage} & \begin{minipage}[b]{0.12\columnwidth}\centering
graisse\strut
\end{minipage} & \begin{minipage}[b]{0.12\columnwidth}\centering
fibre\strut
\end{minipage}\tabularnewline
\midrule
\endfirsthead
\toprule
\begin{minipage}[b]{0.15\columnwidth}\centering
moisEnqu\_e\strut
\end{minipage} & \begin{minipage}[b]{0.17\columnwidth}\centering
consommation\strut
\end{minipage} & \begin{minipage}[b]{0.10\columnwidth}\centering
taille\strut
\end{minipage} & \begin{minipage}[b]{0.09\columnwidth}\centering
poids\strut
\end{minipage} & \begin{minipage}[b]{0.12\columnwidth}\centering
graisse\strut
\end{minipage} & \begin{minipage}[b]{0.12\columnwidth}\centering
fibre\strut
\end{minipage}\tabularnewline
\midrule
\endhead
\begin{minipage}[t]{0.15\columnwidth}\centering
1\strut
\end{minipage} & \begin{minipage}[t]{0.17\columnwidth}\centering
22.86\strut
\end{minipage} & \begin{minipage}[t]{0.10\columnwidth}\centering
181.6\strut
\end{minipage} & \begin{minipage}[t]{0.09\columnwidth}\centering
88.18\strut
\end{minipage} & \begin{minipage}[t]{0.12\columnwidth}\centering
9.168\strut
\end{minipage} & \begin{minipage}[t]{0.12\columnwidth}\centering
1.4\strut
\end{minipage}\tabularnewline
\begin{minipage}[t]{0.15\columnwidth}\centering
7\strut
\end{minipage} & \begin{minipage}[t]{0.17\columnwidth}\centering
23.88\strut
\end{minipage} & \begin{minipage}[t]{0.10\columnwidth}\centering
166\strut
\end{minipage} & \begin{minipage}[t]{0.09\columnwidth}\centering
58.74\strut
\end{minipage} & \begin{minipage}[t]{0.12\columnwidth}\centering
9.651\strut
\end{minipage} & \begin{minipage}[t]{0.12\columnwidth}\centering
0.935\strut
\end{minipage}\tabularnewline
\begin{minipage}[t]{0.15\columnwidth}\centering
3\strut
\end{minipage} & \begin{minipage}[t]{0.17\columnwidth}\centering
24.95\strut
\end{minipage} & \begin{minipage}[t]{0.10\columnwidth}\centering
152.4\strut
\end{minipage} & \begin{minipage}[t]{0.09\columnwidth}\centering
49.9\strut
\end{minipage} & \begin{minipage}[t]{0.12\columnwidth}\centering
11.25\strut
\end{minipage} & \begin{minipage}[t]{0.12\columnwidth}\centering
1.248\strut
\end{minipage}\tabularnewline
\begin{minipage}[t]{0.15\columnwidth}\centering
5\strut
\end{minipage} & \begin{minipage}[t]{0.17\columnwidth}\centering
22.24\strut
\end{minipage} & \begin{minipage}[t]{0.10\columnwidth}\centering
171.2\strut
\end{minipage} & \begin{minipage}[t]{0.09\columnwidth}\centering
89.4\strut
\end{minipage} & \begin{minipage}[t]{0.12\columnwidth}\centering
7.578\strut
\end{minipage} & \begin{minipage}[t]{0.12\columnwidth}\centering
1.557\strut
\end{minipage}\tabularnewline
\begin{minipage}[t]{0.15\columnwidth}\centering
3\strut
\end{minipage} & \begin{minipage}[t]{0.17\columnwidth}\centering
18.54\strut
\end{minipage} & \begin{minipage}[t]{0.10\columnwidth}\centering
177.8\strut
\end{minipage} & \begin{minipage}[t]{0.09\columnwidth}\centering
97.07\strut
\end{minipage} & \begin{minipage}[t]{0.12\columnwidth}\centering
9.147\strut
\end{minipage} & \begin{minipage}[t]{0.12\columnwidth}\centering
0.991\strut
\end{minipage}\tabularnewline
\begin{minipage}[t]{0.15\columnwidth}\centering
3\strut
\end{minipage} & \begin{minipage}[t]{0.17\columnwidth}\centering
20.31\strut
\end{minipage} & \begin{minipage}[t]{0.10\columnwidth}\centering
175.3\strut
\end{minipage} & \begin{minipage}[t]{0.09\columnwidth}\centering
61.01\strut
\end{minipage} & \begin{minipage}[t]{0.12\columnwidth}\centering
8.536\strut
\end{minipage} & \begin{minipage}[t]{0.12\columnwidth}\centering
0.765\strut
\end{minipage}\tabularnewline
\bottomrule
\end{longtable}

\begin{longtable}[]{@{}ccc@{}}
\toprule
\begin{minipage}[b]{0.24\columnwidth}\centering
hauteConsomation\strut
\end{minipage} & \begin{minipage}[b]{0.10\columnwidth}\centering
imc\strut
\end{minipage} & \begin{minipage}[b]{0.10\columnwidth}\centering
MCC\strut
\end{minipage}\tabularnewline
\midrule
\endhead
\begin{minipage}[t]{0.24\columnwidth}\centering
\textless=2750 KCals\strut
\end{minipage} & \begin{minipage}[t]{0.10\columnwidth}\centering
26.74\strut
\end{minipage} & \begin{minipage}[t]{0.10\columnwidth}\centering
0\strut
\end{minipage}\tabularnewline
\begin{minipage}[t]{0.24\columnwidth}\centering
\textless=2750 KCals\strut
\end{minipage} & \begin{minipage}[t]{0.10\columnwidth}\centering
21.32\strut
\end{minipage} & \begin{minipage}[t]{0.10\columnwidth}\centering
0\strut
\end{minipage}\tabularnewline
\begin{minipage}[t]{0.24\columnwidth}\centering
\textless=2750 KCals\strut
\end{minipage} & \begin{minipage}[t]{0.10\columnwidth}\centering
21.48\strut
\end{minipage} & \begin{minipage}[t]{0.10\columnwidth}\centering
1\strut
\end{minipage}\tabularnewline
\begin{minipage}[t]{0.24\columnwidth}\centering
\textless=2750 KCals\strut
\end{minipage} & \begin{minipage}[t]{0.10\columnwidth}\centering
30.51\strut
\end{minipage} & \begin{minipage}[t]{0.10\columnwidth}\centering
1\strut
\end{minipage}\tabularnewline
\begin{minipage}[t]{0.24\columnwidth}\centering
\textless=2750 KCals\strut
\end{minipage} & \begin{minipage}[t]{0.10\columnwidth}\centering
30.71\strut
\end{minipage} & \begin{minipage}[t]{0.10\columnwidth}\centering
1\strut
\end{minipage}\tabularnewline
\begin{minipage}[t]{0.24\columnwidth}\centering
\textless=2750 KCals\strut
\end{minipage} & \begin{minipage}[t]{0.10\columnwidth}\centering
19.86\strut
\end{minipage} & \begin{minipage}[t]{0.10\columnwidth}\centering
1\strut
\end{minipage}\tabularnewline
\bottomrule
\end{longtable}

Remarquons que cet ensemble d'individu est bien une cohorte car nous
avons relevé certaines covariables et les trois données fondamentales
qui sont, la date d'entrée dans l'étude, la date de sortie dans l'étude
et le cause de sortie dans l'étude. Nous pouvons ajouter que les
covariables utilisées dans l'étude sont fixe.

\newpage

\hypertarget{statistique-descriptive}{%
\section{Statistique descriptive}\label{statistique-descriptive}}

Notre jeu de donnée présente 5 variables quantitatives et 9 variables
qualitatives.

Faisons un sommaire des variables quantitatives :

\begin{longtable}[]{@{}ccccc@{}}
\toprule
\begin{minipage}[b]{0.17\columnwidth}\centering
consommation\strut
\end{minipage} & \begin{minipage}[b]{0.17\columnwidth}\centering
fibre\strut
\end{minipage} & \begin{minipage}[b]{0.17\columnwidth}\centering
graisse\strut
\end{minipage} & \begin{minipage}[b]{0.17\columnwidth}\centering
taille\strut
\end{minipage} & \begin{minipage}[b]{0.18\columnwidth}\centering
poids\strut
\end{minipage}\tabularnewline
\midrule
\endhead
\begin{minipage}[t]{0.17\columnwidth}\centering
Min. :17.48\strut
\end{minipage} & \begin{minipage}[t]{0.17\columnwidth}\centering
Min. :0.605\strut
\end{minipage} & \begin{minipage}[t]{0.17\columnwidth}\centering
Min. : 7.26\strut
\end{minipage} & \begin{minipage}[t]{0.17\columnwidth}\centering
Min. :152.4\strut
\end{minipage} & \begin{minipage}[t]{0.18\columnwidth}\centering
Min. : 46.72\strut
\end{minipage}\tabularnewline
\begin{minipage}[t]{0.17\columnwidth}\centering
1st Qu.:25.46\strut
\end{minipage} & \begin{minipage}[t]{0.17\columnwidth}\centering
1st Qu.:1.367\strut
\end{minipage} & \begin{minipage}[t]{0.17\columnwidth}\centering
1st Qu.:11.15\strut
\end{minipage} & \begin{minipage}[t]{0.17\columnwidth}\centering
1st Qu.:168.9\strut
\end{minipage} & \begin{minipage}[t]{0.18\columnwidth}\centering
1st Qu.: 64.64\strut
\end{minipage}\tabularnewline
\begin{minipage}[t]{0.17\columnwidth}\centering
Median :28.11\strut
\end{minipage} & \begin{minipage}[t]{0.17\columnwidth}\centering
Median :1.679\strut
\end{minipage} & \begin{minipage}[t]{0.17\columnwidth}\centering
Median :12.60\strut
\end{minipage} & \begin{minipage}[t]{0.17\columnwidth}\centering
Median :173.0\strut
\end{minipage} & \begin{minipage}[t]{0.18\columnwidth}\centering
Median : 72.80\strut
\end{minipage}\tabularnewline
\begin{minipage}[t]{0.17\columnwidth}\centering
Mean :28.35\strut
\end{minipage} & \begin{minipage}[t]{0.17\columnwidth}\centering
Mean :1.723\strut
\end{minipage} & \begin{minipage}[t]{0.17\columnwidth}\centering
Mean :12.76\strut
\end{minipage} & \begin{minipage}[t]{0.17\columnwidth}\centering
Mean :173.4\strut
\end{minipage} & \begin{minipage}[t]{0.18\columnwidth}\centering
Mean : 72.40\strut
\end{minipage}\tabularnewline
\begin{minipage}[t]{0.17\columnwidth}\centering
3rd Qu.:31.10\strut
\end{minipage} & \begin{minipage}[t]{0.17\columnwidth}\centering
3rd Qu.:1.939\strut
\end{minipage} & \begin{minipage}[t]{0.17\columnwidth}\centering
3rd Qu.:14.02\strut
\end{minipage} & \begin{minipage}[t]{0.17\columnwidth}\centering
3rd Qu.:177.8\strut
\end{minipage} & \begin{minipage}[t]{0.18\columnwidth}\centering
3rd Qu.: 79.44\strut
\end{minipage}\tabularnewline
\begin{minipage}[t]{0.17\columnwidth}\centering
Max. :43.96\strut
\end{minipage} & \begin{minipage}[t]{0.17\columnwidth}\centering
Max. :5.351\strut
\end{minipage} & \begin{minipage}[t]{0.17\columnwidth}\centering
Max. :21.63\strut
\end{minipage} & \begin{minipage}[t]{0.17\columnwidth}\centering
Max. :190.5\strut
\end{minipage} & \begin{minipage}[t]{0.18\columnwidth}\centering
Max. :106.14\strut
\end{minipage}\tabularnewline
\bottomrule
\end{longtable}

Les individus mangent en moyenne 2835 calories par jours. Ils ingèrent
de plus 12.76 grammes de gras par jour, mesurent 173 cm et pèsent 72.40
kilo-gramme en moyenne.\\

Faisons maintenant un sommaire des variables quantitatives :

\begin{longtable}[]{@{}cccc@{}}
\toprule
\begin{minipage}[b]{0.17\columnwidth}\centering
statut\strut
\end{minipage} & \begin{minipage}[b]{0.22\columnwidth}\centering
emploi\strut
\end{minipage} & \begin{minipage}[b]{0.17\columnwidth}\centering
moisEnqu\_e\strut
\end{minipage} & \begin{minipage}[b]{0.23\columnwidth}\centering
hauteConsomation\strut
\end{minipage}\tabularnewline
\midrule
\endhead
\begin{minipage}[t]{0.17\columnwidth}\centering
0 :252\strut
\end{minipage} & \begin{minipage}[t]{0.22\columnwidth}\centering
Bank worker:147\strut
\end{minipage} & \begin{minipage}[t]{0.17\columnwidth}\centering
11 : 39\strut
\end{minipage} & \begin{minipage}[t]{0.23\columnwidth}\centering
\textless=2750 KCals:149\strut
\end{minipage}\tabularnewline
\begin{minipage}[t]{0.17\columnwidth}\centering
3 : 18\strut
\end{minipage} & \begin{minipage}[t]{0.22\columnwidth}\centering
Conductor : 83\strut
\end{minipage} & \begin{minipage}[t]{0.17\columnwidth}\centering
1 : 37\strut
\end{minipage} & \begin{minipage}[t]{0.23\columnwidth}\centering
\textgreater2750 KCals :179\strut
\end{minipage}\tabularnewline
\begin{minipage}[t]{0.17\columnwidth}\centering
13 : 18\strut
\end{minipage} & \begin{minipage}[t]{0.22\columnwidth}\centering
Driver : 98\strut
\end{minipage} & \begin{minipage}[t]{0.17\columnwidth}\centering
3 : 37\strut
\end{minipage} & \begin{minipage}[t]{0.23\columnwidth}\centering
NA\strut
\end{minipage}\tabularnewline
\begin{minipage}[t]{0.17\columnwidth}\centering
12 : 12\strut
\end{minipage} & \begin{minipage}[t]{0.22\columnwidth}\centering
NA\strut
\end{minipage} & \begin{minipage}[t]{0.17\columnwidth}\centering
2 : 34\strut
\end{minipage} & \begin{minipage}[t]{0.23\columnwidth}\centering
NA\strut
\end{minipage}\tabularnewline
\begin{minipage}[t]{0.17\columnwidth}\centering
5 : 10\strut
\end{minipage} & \begin{minipage}[t]{0.22\columnwidth}\centering
NA\strut
\end{minipage} & \begin{minipage}[t]{0.17\columnwidth}\centering
5 : 34\strut
\end{minipage} & \begin{minipage}[t]{0.23\columnwidth}\centering
NA\strut
\end{minipage}\tabularnewline
\begin{minipage}[t]{0.17\columnwidth}\centering
1 : 8\strut
\end{minipage} & \begin{minipage}[t]{0.22\columnwidth}\centering
NA\strut
\end{minipage} & \begin{minipage}[t]{0.17\columnwidth}\centering
12 : 33\strut
\end{minipage} & \begin{minipage}[t]{0.23\columnwidth}\centering
NA\strut
\end{minipage}\tabularnewline
\begin{minipage}[t]{0.17\columnwidth}\centering
(Other): 10\strut
\end{minipage} & \begin{minipage}[t]{0.22\columnwidth}\centering
NA\strut
\end{minipage} & \begin{minipage}[t]{0.17\columnwidth}\centering
(Other):114\strut
\end{minipage} & \begin{minipage}[t]{0.23\columnwidth}\centering
NA\strut
\end{minipage}\tabularnewline
\bottomrule
\end{longtable}

Il y a 2 fois plus de \(\textbf{Bank worker}\) que de
\(\textbf{Conductor}\) ou de \(\textbf{Driver}\). Il y a \(149\)
personnes qui mangent moins de \(2750\) calories par jours et \(179\)
personnes qui en mangent plus.\\

Regardons désormais la corrélation entre les variables quantitatives :\\

\begin{center}\includegraphics{mark_files/figure-latex/unnamed-chunk-5-1} \end{center}

Nous observons que toutes les variables sont corrélées positivement. ~

La variable \textbf{consommation} est très corrélée avec la variable
\textbf{graisse} mais elle est très peu corrélée avec la variable
\textbf{taille}. La variable \textbf{taille} est peu corrélée avec la
variable \textbf{graisse}.\\

Nous allons maintenant regarder le lien entre la maladie et le secteur
d'activité.\\

\begin{center}\includegraphics{mark_files/figure-latex/unnamed-chunk-6-1} \end{center}

Nous observons que la proportion de conducteur est plus élevée chez les
malades que chez les non malades.

Nous allons maintenant regarder le lien entre l'IMC et les maladies
coronariennes.

\begin{center}\includegraphics{mark_files/figure-latex/unnamed-chunk-7-1} \end{center}

Nous remarquons que la mediane de l'IMC du groupe de personnes malades
est à peine plus basse que la médiane pour le groupe de personnes non
malade. Une simple analyse descriptive ne suffit pas pour obtenir des
résultats bien concluants, nous allons continuer avec des méthodes plus
poussées. Nous commencerons par des régréssions logistiques.

\hypertarget{etude-de-la-consommation-de-fibre}{%
\section{Etude de la consommation de
fibre}\label{etude-de-la-consommation-de-fibre}}

Nous allons commencer par une régréssion logistique qui nous permettera
d'expliquer la variable \(\textbf{MCC}\) en fonction de certaines
covariables. Rappelons que dans la régréssion logistique ce n'est pas la
réponse binaire qui est modélisé mais la probabilité de réalisation
d'une des deux modalités (avoir une maladie coronarienne ou non).\\

Nous allons commencer par regarder le meilleur modèle. Nous enlevons la
\(\textbf{taille}\) et le \(\textbf{poids}\) pour laisser
\(\textbf{imc}\) car elles sont très fortement corrélées. Nous avons
tout d'abord voulu comparer en comparant les AIC avec stepAIC.

\begin{Shaded}
\begin{Highlighting}[]
\NormalTok{coeurlogbin<-coeur}\OperatorTok\NormalTok{dplyr}\OperatorTok{::}\KeywordTok{select}\NormalTok{(}\OperatorTok{-}\NormalTok{id,}\OperatorTok{-}\NormalTok{dateEntree,}\OperatorTok{-}\NormalTok{dateSortie,}\OperatorTok{-}\NormalTok{dateNaissance,}\OperatorTok{-}\NormalTok{statut,}
                                   \OperatorTok{-}\NormalTok{poids,}\OperatorTok{-}\NormalTok{taille)}
\NormalTok{res<-}\KeywordTok{glm}\NormalTok{(MCC}\OperatorTok{~}\NormalTok{.,}\DataTypeTok{family =} \KeywordTok{binomial}\NormalTok{(logit),}\DataTypeTok{data=}\NormalTok{coeurlogbin)}
\KeywordTok{stepAIC}\NormalTok{(res)}
\end{Highlighting}
\end{Shaded}

Le modèle ayant le plus faible AIC est le suivant :
\[y=\mu +  \beta_1 fibre\]\\

Nous avons également utilisé les p-values pour trouver le meilleur
modèle en enlevant à la main petit à petit les variables les moins
significatives et nous obtenons encore le même modèle.

\begin{Shaded}
\begin{Highlighting}[]
\NormalTok{coeurlogbin<-coeur}\OperatorTok\NormalTok{dplyr}\OperatorTok{::}\KeywordTok{select}\NormalTok{(}\OperatorTok{-}\NormalTok{id,}\OperatorTok{-}\NormalTok{dateEntree,}\OperatorTok{-}\NormalTok{dateSortie,}\OperatorTok{-}\NormalTok{dateNaissance,}\OperatorTok{-}\NormalTok{statut,}
                                   \OperatorTok{-}\NormalTok{poids,}\OperatorTok{-}\NormalTok{taille)}
\NormalTok{res<-}\KeywordTok{glm}\NormalTok{(MCC}\OperatorTok{~}\NormalTok{.}\OperatorTok{-}\NormalTok{emploi}\OperatorTok{-}\NormalTok{moisEnqu_e}\OperatorTok{-}\NormalTok{consommation}\DecValTok{-1}\OperatorTok{-}\NormalTok{hauteConsomation}\OperatorTok{-}\NormalTok{imc}\OperatorTok{-}\NormalTok{graisse,}
         \DataTypeTok{family =} \KeywordTok{binomial}\NormalTok{(logit),}\DataTypeTok{data=}\NormalTok{coeurlogbin)}
\KeywordTok{summary}\NormalTok{(res)}
\end{Highlighting}
\end{Shaded}

Regardons donc la relation entre les maladies coronariennes et la
consommation de fibre.\\

\begin{Shaded}
\begin{Highlighting}[]
\KeywordTok{ggplot}\NormalTok{(coeurlogbin,}\KeywordTok{aes}\NormalTok{(}\DataTypeTok{x=}\NormalTok{coeurlogbin}\OperatorTok{$}\NormalTok{fibre,}\DataTypeTok{y=}\NormalTok{coeurlogbin}\OperatorTok{$}\NormalTok{MCC))}\OperatorTok{+}\KeywordTok{geom_point}\NormalTok{()}
\end{Highlighting}
\end{Shaded}

\includegraphics{mark_files/figure-latex/unnamed-chunk-10-1.pdf}

Nous observons que les personnes mangeant plus de 2g de fibre par jour
sont moins malades.\\

Nous voulons faire une régression logistique, nous allons donc vérifier
les conditions d'application. Il est recommandé d'avoir en pratique 10
fois plus d'évenements que de paramètres dans le modèle. Nous allons
utiliser ici 2 paramétres ( en comptant l'intercept) nous devrions donc
avoir au moins 20 malades.

\begin{Shaded}
\begin{Highlighting}[]
\KeywordTok{table}\NormalTok{(coeurlogbin}\OperatorTok{$}\NormalTok{MCC)}
\end{Highlighting}
\end{Shaded}

\begin{verbatim}
## 
##   0   1 
## 252  76
\end{verbatim}

Nous avons 76 malades nous pouvons donc continuer.\\

Il faut maintenant vérifier que nous ne sommes pas dans le cas de
surdipersion c'est à dire qu'il ne faut pas que la dispersion réelle des
données soit supérieure à celle prévue par la théorie car dans ce cas
l'erreur standard des paramètres est sous-estimée ce qui peut conduire à
des p-valeurs très faible et donner des conclusions erronnées. Evaluons
donc s'il y a ou non une surdispersion :

\begin{Shaded}
\begin{Highlighting}[]
\NormalTok{reslog<-}\KeywordTok{glm}\NormalTok{(MCC}\OperatorTok{~}\NormalTok{fibre,}\DataTypeTok{family =} \KeywordTok{binomial}\NormalTok{(logit),}\DataTypeTok{data=}\NormalTok{coeurlogbin)}
\KeywordTok{summary}\NormalTok{(reslog)}
\end{Highlighting}
\end{Shaded}

\begin{verbatim}
## 
## Call:
## glm(formula = MCC ~ fibre, family = binomial(logit), data = coeurlogbin)
## 
## Deviance Residuals: 
##     Min       1Q   Median       3Q      Max  
## -0.9166  -0.7624  -0.7002  -0.5372   1.9817  
## 
## Coefficients:
##             Estimate Std. Error z value Pr(>|z|)  
## (Intercept)  -0.3413     0.4588  -0.744   0.4569  
## fibre        -0.5101     0.2678  -1.905   0.0568 .
## ---
## Signif. codes:  0 '***' 0.001 '**' 0.01 '*' 0.05 '.' 0.1 ' ' 1
## 
## (Dispersion parameter for binomial family taken to be 1)
## 
##     Null deviance: 355.11  on 327  degrees of freedom
## Residual deviance: 351.12  on 326  degrees of freedom
## AIC: 355.12
## 
## Number of Fisher Scoring iterations: 4
\end{verbatim}

\[\frac{deviance residuelle}{nddl}=\frac{351.12}{326}=1.08\] nous
pouvons ainsi considérer qu'il n'y a pas surdispersion.\\

Nous pouvons donc maintenant utiliser la régression logistique et faire
des interprétations :\\

\begin{Shaded}
\begin{Highlighting}[]
\NormalTok{reslog<-}\KeywordTok{glm}\NormalTok{(MCC}\OperatorTok{~}\NormalTok{fibre}\DecValTok{-1}\NormalTok{,}\DataTypeTok{family =} \KeywordTok{binomial}\NormalTok{(logit),}\DataTypeTok{data=}\NormalTok{coeurlogbin)}
\KeywordTok{summary}\NormalTok{(reslog)}
\end{Highlighting}
\end{Shaded}

\begin{verbatim}
## 
## Call:
## glm(formula = MCC ~ fibre - 1, family = binomial(logit), data = coeurlogbin)
## 
## Deviance Residuals: 
##     Min       1Q   Median       3Q      Max  
## -1.0030  -0.7823  -0.6959  -0.4816   2.0741  
## 
## Coefficients:
##       Estimate Std. Error z value Pr(>|z|)    
## fibre -0.70288    0.07799  -9.013   <2e-16 ***
## ---
## Signif. codes:  0 '***' 0.001 '**' 0.01 '*' 0.05 '.' 0.1 ' ' 1
## 
## (Dispersion parameter for binomial family taken to be 1)
## 
##     Null deviance: 454.70  on 328  degrees of freedom
## Residual deviance: 351.66  on 327  degrees of freedom
## AIC: 353.66
## 
## Number of Fisher Scoring iterations: 4
\end{verbatim}

\begin{Shaded}
\begin{Highlighting}[]
\KeywordTok{confint}\NormalTok{(reslog)}
\end{Highlighting}
\end{Shaded}

\begin{verbatim}
##      2.5 %     97.5 % 
## -0.8604690 -0.5542772
\end{verbatim}

\[ln \bigg(\frac{odds(Y =1  | fibre = x_1+1)}{odds(Y=1 | fibre = x_1)}\bigg)=\hat{\beta}_{fibre}=-0.70288\]
Augmenter la consommation de fibre de 1g par jour va multiplier la
chance d'avoir une maladie coronarienne par au moins
\(\exp(-0.8604690)=0.43\) et au plus \(\exp(-0.5542772)=0.57\).
Autrement dit, si nous augmentons notre consommation de fibre de 1g par
jour nous divisons par au moins (1/0.43)=2.32 l'odds de contracter une
maladie coronarienne.\\
Nous pouvons alors conseiller de manger des fibres si nous voulons nous
prévenir des maladies coronariennes.

\hypertarget{etude-de-la-consommation}{%
\section{Etude de la consommation}\label{etude-de-la-consommation}}

Nous voulons maintenant étudier la consommation à l'aide d'une
régression polytomique ordonné. En effet nous aimerions voir ce qui
influence la consommation de calories des personnes. Nous Nous voulons
au moins 3 modalités, nous recodons donc la variable
\(\textbf{consommation}\) pour qu'elle soit qualitative à 3 modalités.
Nous classons ainsi la consommation en ``faible'', ``moyen'' et
``élevée''. Nous avons alors bien une relation d'ordre
\("faible" < "moyen" < "élevée"\) .\\

\begin{Shaded}
\begin{Highlighting}[]
\NormalTok{coeur <-}\KeywordTok{mutate}\NormalTok{(coeur,}\DataTypeTok{consom_dec =}\KeywordTok{case_when}\NormalTok{(}
\NormalTok{  consommation}\OperatorTok{<}\DecValTok{23}\OperatorTok{~}\StringTok{"faible"}\NormalTok{,}
\NormalTok{  consommation}\OperatorTok{>}\DecValTok{22} \OperatorTok{&}\StringTok{ }\NormalTok{consommation}\OperatorTok{<}\DecValTok{30}\OperatorTok{~}\StringTok{"moyen"}\NormalTok{,}
\NormalTok{  consommation}\OperatorTok{>}\DecValTok{29} \OperatorTok{~}\StringTok{"élevée"}\NormalTok{))}
\NormalTok{coeur}\OperatorTok{$}\NormalTok{consom_dec<-}\KeywordTok{as.factor}\NormalTok{(coeur}\OperatorTok{$}\NormalTok{consom_dec)}
\end{Highlighting}
\end{Shaded}

Nous faisons un stepAIC pour choisir le meilleur modèle au sens du
critère \(\textbf{AIC}\).\\

\begin{Shaded}
\begin{Highlighting}[]
\NormalTok{coeurlogbin1<-coeur}\OperatorTok\NormalTok{dplyr}\OperatorTok{::}\KeywordTok{select}\NormalTok{(}\OperatorTok{-}\NormalTok{id,}\OperatorTok{-}\NormalTok{dateEntree,}\OperatorTok{-}\NormalTok{dateSortie,}\OperatorTok{-}\NormalTok{dateNaissance,}\OperatorTok{-}\NormalTok{statut,}
                                    \OperatorTok{-}\NormalTok{hauteConsomation,}\OperatorTok{-}\NormalTok{consommation)}
\NormalTok{modele<-}\KeywordTok{polr}\NormalTok{(consom_dec}\OperatorTok{~}\NormalTok{.,}\DataTypeTok{data=}\NormalTok{coeurlogbin1)}
\KeywordTok{stepAIC}\NormalTok{(modele)}
\end{Highlighting}
\end{Shaded}

Nous obtenons qu'il faut garder \(\textbf{graisse}\) et
\(\textbf{fibre}\), nous faisons donc la régréssion polytomique ordonné
selon ces variables.\\

\begin{Shaded}
\begin{Highlighting}[]
\NormalTok{modele2<-}\KeywordTok{polr}\NormalTok{(consom_dec}\OperatorTok{~}\NormalTok{graisse}\OperatorTok{+}\NormalTok{fibre,}\DataTypeTok{data=}\NormalTok{coeurlogbin1)}
\NormalTok{modele2}
\end{Highlighting}
\end{Shaded}

\begin{verbatim}
## Call:
## polr(formula = consom_dec ~ graisse + fibre, data = coeurlogbin1)
## 
## Coefficients:
##    graisse      fibre 
## -0.4308596 -0.5959642 
## 
## Intercepts:
## élevée|faible  faible|moyen 
##     -7.240439     -6.698937 
## 
## Residual Deviance: 504.4718 
## AIC: 512.4718
\end{verbatim}

\begin{Shaded}
\begin{Highlighting}[]
\KeywordTok{confint}\NormalTok{(modele2)}
\end{Highlighting}
\end{Shaded}

\begin{verbatim}
##             2.5 %     97.5 %
## graisse -0.559942 -0.3106530
## fibre   -1.099625 -0.1207172
\end{verbatim}

Notons que la commande \(polyr\) utilisée renvoie l'opposé du
coefficient \(\beta\) considéré donc nous obtenons les ``vrais''
résultats suivants : \(\hat{\beta}_{graisse}=0.43\) et
\(\hat{\beta_{fibre}}=0.59\) avec les intervalles de confiance suivant :
\(\hat{\beta}_{graisse} \in [0.31,0.56]\) et
\(\hat{\beta}_{fibre} \in [0.12,1.10]\)

Nous pouvons regarder noter que :
\[ln \bigg(\frac{odds(Y \leq moyen | graisse = x_1+1, fibre=x_2)}{odds(Y \leq moyen | graisse = x_1, fibre=x_2)}\bigg)=\hat{\beta}_{graisse}\]

A consommation de fibre fixée, augmenter la consommation de graisse de 1
g/jour va multiplier l'odds de \(Y \leq moyen\) par au moins
\(exp(0.31)=1.36\) et au plus \(exp(0.56)=1.75\). ~

De plus :\\
\[ln \bigg(\frac{odds(Y \leq moyen | fibre = x_1+1, graisse=x_2)}{odds(Y \leq moyen | fibre = x_1, graisse=x_2)}\bigg)=\hat{\beta}_{fibre}\]

A consommation de graisse fixée, augmenter la consommation de fibre de 1
g/jour va multiplier l'odds de \(Y \leq moyen\) par au moins
\(exp(0.12)=1.12\) et au plus \(exp(1.10)=3\). Ainsi manger plus de
fibre va augmenter les chances que la personne mange moins de calorie
dans la journée. Ceci n'est pas surpenant car les fibres sont connus
pour être rassasiant.\\

Nous avons vu précedemment qu'augmenter sa consommation de fibre était
une bonne chose pour se prévenir des maladies coronariennes, nous savons
maintenant que à consommation de graisse fixée la consommation de fibre
permet également au personne de ne pas manger trop de calories.\\

\hypertarget{uxe9tude-de-lemploi}{%
\section{Étude de l'emploi}\label{uxe9tude-de-lemploi}}

Nous voulons maintenant étudier le l'emploi à l'aide d'une régression
polytomique non ordonné. En effet nous aimerions voir ce qui influence
la préférence de choisir un type d'emploi par apport aux autres. La
variable emploi est composée de 3 modalités, qui sont : \(conductor\),
\(bank worker\) et \(driver\).

Nous faisons un stepAIC pour choisir le meilleur modèle au sens du
critère AIC

Nous obtenons qu'il faut garder taille et poids. Nous faisons donc la
régréssion polytomique non ordonné selon ces variables.

\begin{Shaded}
\begin{Highlighting}[]
\NormalTok{modele4<-}\KeywordTok{multinom}\NormalTok{(emploi}\OperatorTok{~}\NormalTok{poids}\OperatorTok{+}\NormalTok{taille,}\DataTypeTok{data=}\NormalTok{coeurlogbin1)}
\end{Highlighting}
\end{Shaded}

\begin{verbatim}
## # weights:  12 (6 variable)
## initial  value 360.344831 
## iter  10 value 302.773746
## iter  20 value 301.538107
## final  value 301.526738 
## converged
\end{verbatim}

\begin{Shaded}
\begin{Highlighting}[]
\NormalTok{modele4}
\end{Highlighting}
\end{Shaded}

\begin{verbatim}
## Call:
## multinom(formula = emploi ~ poids + taille, data = coeurlogbin1)
## 
## Coefficients:
##           (Intercept)       poids     taille
## Conductor    34.13104 -0.07620753 -0.1694273
## Driver       22.11401 -0.01424113 -0.1231291
## 
## Residual Deviance: 603.0535 
## AIC: 615.0535
\end{verbatim}

\begin{Shaded}
\begin{Highlighting}[]
\KeywordTok{confint}\NormalTok{(modele4)}
\end{Highlighting}
\end{Shaded}

\begin{verbatim}
## , , Conductor
## 
##                  2.5 %      97.5 %
## (Intercept) 30.3445677 37.91750699
## poids       -0.1114239 -0.04099117
## taille      -0.1959546 -0.14290007
## 
## , , Driver
## 
##                   2.5 %      97.5 %
## (Intercept) 16.95239383 27.27563010
## poids       -0.04357888  0.01509662
## taille      -0.15673368 -0.08952445
\end{verbatim}

Bank worker est la modalité de référence pour emploi.\\

\[\hat{\beta}_{poids|conductor}=ln \bigg(\frac{\frac{P(conductor| poids = x_1+1, taille=x_2)}{P(Bank worker | poids = x_1+1, taille=x_2)}}{\frac{P(conductor | poids = x_1, taille=x_2)}{P(Bank worker | poids = x_1,taille=x_2)}}\bigg)=-0.07620753 \]

À taille fixé, un poids de 1kg en plus va multiplier par au moins
\(exp(-0.1114239)=0.8945595\) et au plus\(exp( -0.04099117)=0.9598376\)
la préférence de conductor par rapport à Bank worker.\\

\[\hat{\beta}_{taille|conductor}=ln \bigg(\frac{\frac{P(conductor| taille = x_1+1, poids=x_2)}{P(Bank worker | taille = x_1+1, poids=x_2)}}{\frac{P(conductor | taille = x_1, poids=x_2)}{P(Bank worker | taille = x_1,poids=x_2)}}\bigg)=-0.1694273\]

À poids fixé, une taille qui augmente de 1cm va multiplier par au moins
\(exp( -0.1959546)=\) et au plus \(exp( -0.14290007)=0.8668407\) la
préférence de conductor par rapport à Bank worker.\\

\[\hat{\beta}_{taille|driver}=ln \bigg(\frac{\frac{P(driver| taille = x_1+1, poids=x_2)}{P(Bank worker | taille = x_1+1, poids=x_2)}}{\frac{P(driver | taille = x_1, poids=x_2)}{P(Bank worker | taille = x_1,poids=x_2)}}\bigg)=-0.1231291\]

À poids fixé, une taille qui augmente de 1cm va multiplier par au moins
\(exp( -0.15673368)=0.8549317\) et au plus
\(exp( -0.08952445)=0.9143659\) la préférence de driver par rapport à
Bank worker.\\

Ainsi, la taille et le poids ont un effet sur le métier que l'on va
choisir.

\hypertarget{uxe9tude-de-lincidence-instantannuxe9e-de-la-maladie}{%
\section{Étude de l'incidence instantannée de la
maladie}\label{uxe9tude-de-lincidence-instantannuxe9e-de-la-maladie}}

Dans cette partie, nous allons chercher à répondre à la problématique
suivante : à une date donnée, quelle sera le taux de nouveaux malades
dans la population étudiée ?

Nous allons désormais nettoyer la base de données et faire des
transformations de format des variables pour pouvoir utiliser le modèle
de Cox. Les variables qui contiennent une date ( date entrée, date
sortie et date de naissance) sont de type ``charactère'' nous allons
donc dans un premier temps les convertir en type ``Date''.

Ensuite, la fonction qui exécute le modèle de cox à besoin de valeurs
numériques représentant les dates. Sous R, chaque date est repésentée
par un nombre de jour à partir d'une date d'origine : le 1 janvier 1970.
Nous allons donc créer une variable qui récupérera ce nombre pour chaque
date correspondant à chaques individus.

Nous pouvons désormais faire nos analyse avec le modèle de Cox. Pour une
première analyse, nous estimerons notre modèle en prenant en compte
toute les covariables possible présentent dans la table de données. Mais
le modèle de risques proportionnels de Cox fait plusieurs hypothèses.
Ainsi, il est important d'évaluer si un tel modèle ajusté décrit
correctement les données.

Ici, nous allons discuter de trois types de diagonostiques pour le
modèle de Cox:

\hypertarget{duxe9tection-de-la-non-linuxe9arituxe9-des-variables}{%
\subsection{Détection de la non-linéarité des
variables}\label{duxe9tection-de-la-non-linuxe9arituxe9-des-variables}}

Souvent, nous supposons que les covariables continues ont une forme
linéaire. Cependant, cette hypothèse doit être vérifiée. Le traçage des
résidus de Martingale par rapport à des covariables continues est une
approche courante utilisée pour détecter la non - linéarité ou, en
d'autres termes, pour évaluer la forme fonctionnelle d'une covariable.
Pour une covariable continue donnée, les modèles du graphique peuvent
suggérer que la variable n'est pas correctement ajustée.

La non-linéarité n'est pas un problème pour les variables catégorielles,
nous n'examinons donc que les graphiques des résidus de martingale par
rapport à une variable continue. Testons avec le poids :

\begin{Shaded}
\begin{Highlighting}[]
\NormalTok{survie=}\KeywordTok{Surv}\NormalTok{(coeur2}\OperatorTok{$}\NormalTok{date_entree_num,coeur2}\OperatorTok{$}\NormalTok{date_sortie_num,coeur2}\OperatorTok{$}\NormalTok{MCC)}
\NormalTok{res=}\KeywordTok{coxph}\NormalTok{(survie}\OperatorTok{~}\NormalTok{fibre}\OperatorTok{+}\NormalTok{taille}\OperatorTok{+}\NormalTok{poids}\OperatorTok{+}\NormalTok{consommation}\OperatorTok{+}
\StringTok{            }\NormalTok{emploi}\OperatorTok{+}\NormalTok{graisse,}\DataTypeTok{id=}\NormalTok{id,}\DataTypeTok{data=}\NormalTok{coeur2)}
\NormalTok{mresids <-}\StringTok{ }\KeywordTok{residuals}\NormalTok{( res, }\DataTypeTok{type=}\StringTok{"martingale"}\NormalTok{ )}
\NormalTok{lmfit <-}\StringTok{ }\KeywordTok{lm}\NormalTok{(poids}\OperatorTok{~}\NormalTok{taille}\OperatorTok{+}\NormalTok{consommation}\OperatorTok{+}\NormalTok{graisse}\OperatorTok{+}\NormalTok{fibre,}\DataTypeTok{data=}\NormalTok{coeur2 )}
\NormalTok{rbili <-}\StringTok{ }\NormalTok{lmfit}\OperatorTok{$}\NormalTok{resid}
\NormalTok{ord <-}\StringTok{ }\KeywordTok{order}\NormalTok{( rbili )}
\NormalTok{mresids <-}\StringTok{ }\NormalTok{mresids[ ord ]}
\KeywordTok{plot}\NormalTok{( rbili, mresids )}
\KeywordTok{lines}\NormalTok{( }\KeywordTok{smooth.spline}\NormalTok{( rbili, mresids, }\DataTypeTok{df=}\DecValTok{6}\NormalTok{ ), }\DataTypeTok{col=}\StringTok{"red"}\NormalTok{, }\DataTypeTok{lwd=}\DecValTok{2}\NormalTok{ )}
\KeywordTok{lines}\NormalTok{( rbili, }\KeywordTok{fitted}\NormalTok{(}\KeywordTok{lm}\NormalTok{( mresids }\OperatorTok{~}\StringTok{ }\NormalTok{rbili )), }\DataTypeTok{col=}\StringTok{"blue"}\NormalTok{, }\DataTypeTok{lwd=}\DecValTok{2}\NormalTok{ )}
\end{Highlighting}
\end{Shaded}

\begin{center}\includegraphics{mark_files/figure-latex/unnamed-chunk-21-1} \end{center}

Il semble que le poids soit linéaire. Maintenant testons avec la
variable fibre

\begin{center}\includegraphics{mark_files/figure-latex/unnamed-chunk-22-1} \end{center}

On distingue clairement 2 groupes de résidus et la courbe varie un peu.
On va appliquer une transformation logarithme à cette variable :

\begin{center}\includegraphics{mark_files/figure-latex/unnamed-chunk-23-1} \end{center}

On voit que c'est déja un peu mieux. On appliquera le logarithme à fibre
dans nos modèles. Testons la consommation

\begin{center}\includegraphics{mark_files/figure-latex/unnamed-chunk-24-1} \end{center}

Il n'y a pas besoin d'appliquer de transformations, les résidus ne sont
pas clairement entassé comme l'exemple fibre.\\
Testons pour graisse :

\begin{center}\includegraphics{mark_files/figure-latex/unnamed-chunk-25-1} \end{center}

Comme précédemment, nous n'avons pas besoin d'appliquer de
transformation.\\
Testons désormais la taille :

\begin{center}\includegraphics{mark_files/figure-latex/unnamed-chunk-26-1} \end{center}

Essayons de voir ce qu'il se passe lorsque nous appliquons le logarithme
:

\begin{center}\includegraphics{mark_files/figure-latex/unnamed-chunk-27-1} \end{center}

On la courbe parait plus stable. On fera donc une transformation
logarithme sur la variable taille.

\hypertarget{examiner-les-observations-influentes-ou-les-valeurs-aberrantes.}{%
\subsection{Examiner les observations influentes (ou les valeurs
aberrantes).}\label{examiner-les-observations-influentes-ou-les-valeurs-aberrantes.}}

Pour tester des observations influentes ou des valeurs aberrantes, nous
pouvons utiliser les valeurs \(dfbeta\). Le principe est de comparer le
coefficient de une variable donnée lorsque un point donnée participe ou
pas à la régression. On mesure l'influence d'un point sur le coefficient
estimé.

\begin{Shaded}
\begin{Highlighting}[]
\KeywordTok{ggcoxdiagnostics}\NormalTok{(res, }\DataTypeTok{type =} \StringTok{"dfbeta"}\NormalTok{,}
                 \DataTypeTok{linear.predictions =} \OtherTok{FALSE}\NormalTok{, }\DataTypeTok{ggtheme =} \KeywordTok{theme_bw}\NormalTok{())}
\end{Highlighting}
\end{Shaded}

\includegraphics{mark_files/figure-latex/unnamed-chunk-28-1.pdf}

Nous voyons que que la comparaison des valeurs les plus élevées aux
coefficients de régression suggère qu'aucune des observations n'est
terriblement influente individuellement.

\hypertarget{test-de-lhypothuxe8se-des-risques-proportionnels.}{%
\subsection{Test de l'hypothèse des risques
proportionnels.}\label{test-de-lhypothuxe8se-des-risques-proportionnels.}}

Regardons désormais si l'hypothèse d'indépendance du temps des formes
multiplicatives et des covariables est vérifiée. L'hypothèse des risques
proportionnels peut être vérifiée à l'aide de tests statistiques et de
diagnostics graphiques basés sur les résidus de Schoenfeld.

\begin{Shaded}
\begin{Highlighting}[]
\NormalTok{res=}\KeywordTok{coxph}\NormalTok{(survie}\OperatorTok{~}\KeywordTok{log}\NormalTok{(fibre)}\OperatorTok{+}\KeywordTok{log}\NormalTok{(taille)}\OperatorTok{+}\NormalTok{poids}\OperatorTok{+}\NormalTok{consommation}\OperatorTok{+}
\StringTok{            }\NormalTok{emploi}\OperatorTok{+}\NormalTok{graisse,}\DataTypeTok{id=}\NormalTok{id,}\DataTypeTok{data=}\NormalTok{coeur2)}
\NormalTok{res.c=}\KeywordTok{cox.zph}\NormalTok{(res)}
\NormalTok{res.c}
\end{Highlighting}
\end{Shaded}

\begin{verbatim}
##               chisq df     p
## log(fibre)   0.0159  1 0.900
## log(taille)  0.0155  1 0.901
## poids        0.0730  1 0.787
## consommation 2.9173  1 0.088
## emploi       4.2252  2 0.121
## graisse      1.5216  1 0.217
## GLOBAL       9.3482  7 0.229
\end{verbatim}

Le test conduit à ne pas rejeter cette hypothèse au seuil de \(5\%\) :
aucune covariable n'a un effet dépendant du temps.

\begin{Shaded}
\begin{Highlighting}[]
\NormalTok{ res=}\KeywordTok{coxph}\NormalTok{(survie}\OperatorTok{~}\KeywordTok{log}\NormalTok{(fibre)}\OperatorTok{+}\KeywordTok{log}\NormalTok{(taille)}\OperatorTok{+}\NormalTok{poids}\OperatorTok{+}\NormalTok{emploi}\OperatorTok{+}
\StringTok{             }\NormalTok{graisse,}\DataTypeTok{id=}\NormalTok{id,}\DataTypeTok{data=}\NormalTok{coeur2)}
 \KeywordTok{summary}\NormalTok{(res)}
\end{Highlighting}
\end{Shaded}

\begin{verbatim}
## Call:
## coxph(formula = survie ~ log(fibre) + log(taille) + poids + emploi + 
##     graisse, data = coeur2, id = id)
## 
##   n= 328, number of events= 76 
## 
##                       coef  exp(coef)   se(coef)      z Pr(>|z|)  
## log(fibre)      -0.3890266  0.6777163  0.4414874 -0.881   0.3782  
## log(taille)     -8.1322981  0.0002939  3.5218163 -2.309   0.0209 *
## poids            0.0002774  1.0002775  0.0139152  0.020   0.9841  
## emploiConductor  0.0279616  1.0283562  0.3214823  0.087   0.9307  
## emploiDriver    -0.0623337  0.9395693  0.2985412 -0.209   0.8346  
## graisse         -0.0484210  0.9527326  0.0571941 -0.847   0.3972  
## ---
## Signif. codes:  0 '***' 0.001 '**' 0.01 '*' 0.05 '.' 0.1 ' ' 1
## 
##                 exp(coef) exp(-coef) lower .95 upper .95
## log(fibre)      0.6777163     1.4755 2.853e-01    1.6101
## log(taille)     0.0002939  3402.6100 2.954e-07    0.2924
## poids           1.0002775     0.9997 9.734e-01    1.0279
## emploiConductor 1.0283562     0.9724 5.476e-01    1.9310
## emploiDriver    0.9395693     1.0643 5.234e-01    1.6867
## graisse         0.9527326     1.0496 8.517e-01    1.0657
## 
## Concordance= 0.617  (se = 0.031 )
## Likelihood ratio test= 13.15  on 6 df,   p=0.04
## Wald test            = 13.62  on 6 df,   p=0.03
## Score (logrank) test = 13.68  on 6 df,   p=0.03
\end{verbatim}

Le test de Wald teste l'effet d'une covariable avec les autres
covariable dans le modèle. S'il n'est pas significatif, cela ne veut pas
dire qu'il ne le serait pas dans le modèle constitué uniquement de cette
covariable.

Le test de Wald pour la covariable « taille » montre que le coefficient
correspondant est fortement significatifs au seuil \(5\%\)
\(p_{values} < 0.05\). Les autres covariables ne modifient pas
significativement cette incidence lorsque taille est dans le modèle.

Nous sélectionnons les variables pas à pas c'est à dire que nous
enlèvons celle dont la \(p_{valeur}\) est la plus élevée. Ensuite nous
refaisons tourner le modèle et nous recommençons jusqu'à obtention de
toutes les variables significatives.

Nous obtenons le modèle suivant :

\begin{Shaded}
\begin{Highlighting}[]
\NormalTok{res=}\KeywordTok{coxph}\NormalTok{(survie}\OperatorTok{~}\KeywordTok{log}\NormalTok{(taille),}\DataTypeTok{id=}\NormalTok{id,}\DataTypeTok{data=}\NormalTok{coeur2)}
\KeywordTok{summary}\NormalTok{(res)}
\end{Highlighting}
\end{Shaded}

\begin{verbatim}
## Call:
## coxph(formula = survie ~ log(taille), data = coeur2, id = id)
## 
##   n= 328, number of events= 76 
## 
##                   coef  exp(coef)   se(coef)      z Pr(>|z|)   
## log(taille) -9.474e+00  7.685e-05  2.904e+00 -3.262  0.00111 **
## ---
## Signif. codes:  0 '***' 0.001 '**' 0.01 '*' 0.05 '.' 0.1 ' ' 1
## 
##             exp(coef) exp(-coef) lower .95 upper .95
## log(taille) 7.685e-05      13013 2.592e-07   0.02279
## 
## Concordance= 0.588  (se = 0.03 )
## Likelihood ratio test= 10.35  on 1 df,   p=0.001
## Wald test            = 10.64  on 1 df,   p=0.001
## Score (logrank) test = 10.66  on 1 df,   p=0.001
\end{verbatim}

\begin{Shaded}
\begin{Highlighting}[]
\KeywordTok{confint}\NormalTok{(res)}
\end{Highlighting}
\end{Shaded}

\begin{verbatim}
##                 2.5 %    97.5 %
## log(taille) -15.16573 -3.781619
\end{verbatim}

En supposant que toutes les autres covariables sont fixées,
l'augmentation de la taille de 1cm multiplie l'incidence instentannée de
la maladie par au moins, \(exp(-15.16573)=2.591834e-07\) et au plus
\(exp(-3.781619)=0.02278577\).

Auparavant, nous avont constater que dans le modèle constitué de toutes
les covariables, tous les autres coefficients n'étaient pas
significatifs au seuil de 5 pourcents. Mais qu'en est t-il si on test
avec un modèle avec seulement 1 covariable. Regardons ce qui se passe
pour fibre :

\begin{Shaded}
\begin{Highlighting}[]
\NormalTok{res=}\KeywordTok{coxph}\NormalTok{(survie}\OperatorTok{~}\KeywordTok{log}\NormalTok{(fibre),}\DataTypeTok{id=}\NormalTok{id, }\DataTypeTok{data=}\NormalTok{coeur2)}
\KeywordTok{summary}\NormalTok{(res)}
\end{Highlighting}
\end{Shaded}

\begin{verbatim}
## Call:
## coxph(formula = survie ~ log(fibre), data = coeur2, id = id)
## 
##   n= 328, number of events= 76 
## 
##               coef exp(coef) se(coef)      z Pr(>|z|)  
## log(fibre) -0.8061    0.4466   0.3699 -2.179   0.0293 *
## ---
## Signif. codes:  0 '***' 0.001 '**' 0.01 '*' 0.05 '.' 0.1 ' ' 1
## 
##            exp(coef) exp(-coef) lower .95 upper .95
## log(fibre)    0.4466      2.239    0.2163    0.9221
## 
## Concordance= 0.569  (se = 0.032 )
## Likelihood ratio test= 4.71  on 1 df,   p=0.03
## Wald test            = 4.75  on 1 df,   p=0.03
## Score (logrank) test = 4.7  on 1 df,   p=0.03
\end{verbatim}

\begin{Shaded}
\begin{Highlighting}[]
\KeywordTok{confint}\NormalTok{(res)}
\end{Highlighting}
\end{Shaded}

\begin{verbatim}
##                2.5 %      97.5 %
## log(fibre) -1.531084 -0.08111641
\end{verbatim}

« fibre » a un coefficient significatif au seuil 5\% car
\(p_{values} =0.0293 < 0.05\).

En supposant que toutes les autres covariables sont fixées,
l'augmentation de fibre de 1 unité multiplie l'incidence instantannée de
la maladie par au moins \(exp(-1.531084)=0.2163011\).

Testons maintenant pour la variable graisse

\begin{Shaded}
\begin{Highlighting}[]
\NormalTok{ res=}\KeywordTok{coxph}\NormalTok{(survie}\OperatorTok{~}\NormalTok{coeur2}\OperatorTok{$}\NormalTok{graisse,}\DataTypeTok{id=}\NormalTok{coeur2}\OperatorTok{$}\NormalTok{id)}
\KeywordTok{summary}\NormalTok{(res)}
\end{Highlighting}
\end{Shaded}

\begin{verbatim}
## Call:
## coxph(formula = survie ~ coeur2$graisse, id = coeur2$id)
## 
##   n= 328, number of events= 76 
## 
##                    coef exp(coef) se(coef)      z Pr(>|z|)  
## coeur2$graisse -0.09396   0.91032  0.05125 -1.833   0.0668 .
## ---
## Signif. codes:  0 '***' 0.001 '**' 0.01 '*' 0.05 '.' 0.1 ' ' 1
## 
##                exp(coef) exp(-coef) lower .95 upper .95
## coeur2$graisse    0.9103      1.099    0.8233     1.007
## 
## Concordance= 0.564  (se = 0.033 )
## Likelihood ratio test= 3.5  on 1 df,   p=0.06
## Wald test            = 3.36  on 1 df,   p=0.07
## Score (logrank) test = 3.35  on 1 df,   p=0.07
\end{verbatim}

\begin{Shaded}
\begin{Highlighting}[]
\KeywordTok{confint}\NormalTok{(res)}
\end{Highlighting}
\end{Shaded}

\begin{verbatim}
##                     2.5 %      97.5 %
## coeur2$graisse -0.1944183 0.006496239
\end{verbatim}

« graisse » a une \(p_{values}= 0.0668> 0.05\). De plus, l'intervalle de
confiance n'est pas interpretable car il contient 0. Nous ne pouvons pas
faire d'estimation avec ce modèle.

\begin{Shaded}
\begin{Highlighting}[]
\NormalTok{res=}\KeywordTok{coxph}\NormalTok{(survie}\OperatorTok{~}\NormalTok{coeur2}\OperatorTok{$}\NormalTok{emploi,}\DataTypeTok{id=}\NormalTok{coeur2}\OperatorTok{$}\NormalTok{id)}
\KeywordTok{summary}\NormalTok{(res)}
\end{Highlighting}
\end{Shaded}

\begin{verbatim}
## Call:
## coxph(formula = survie ~ coeur2$emploi, id = coeur2$id)
## 
##   n= 328, number of events= 76 
## 
##                          coef exp(coef) se(coef)     z Pr(>|z|)  
## coeur2$emploiConductor 0.4637    1.5899   0.2786 1.664   0.0961 .
## coeur2$emploiDriver    0.2240    1.2511   0.2841 0.788   0.4305  
## ---
## Signif. codes:  0 '***' 0.001 '**' 0.01 '*' 0.05 '.' 0.1 ' ' 1
## 
##                        exp(coef) exp(-coef) lower .95 upper .95
## coeur2$emploiConductor     1.590     0.6290    0.9209     2.745
## coeur2$emploiDriver        1.251     0.7993    0.7168     2.183
## 
## Concordance= 0.548  (se = 0.031 )
## Likelihood ratio test= 2.73  on 2 df,   p=0.3
## Wald test            = 2.78  on 2 df,   p=0.2
## Score (logrank) test = 2.82  on 2 df,   p=0.2
\end{verbatim}

\begin{Shaded}
\begin{Highlighting}[]
\KeywordTok{confint}\NormalTok{(res)}
\end{Highlighting}
\end{Shaded}

\begin{verbatim}
##                              2.5 %    97.5 %
## coeur2$emploiConductor -0.08243578 1.0097674
## coeur2$emploiDriver    -0.33288997 0.7808997
\end{verbatim}

Les intervalles de confiance comprennent 0, rien n'est interpretable.

\hypertarget{conclusion-comment-diminuer-les-rirsques-davoir-une-maladie-coronarienne.}{%
\section{Conclusion : Comment diminuer les rirsques d'avoir une maladie
coronarienne.}\label{conclusion-comment-diminuer-les-rirsques-davoir-une-maladie-coronarienne.}}

Manger des fibres

\end{document}
